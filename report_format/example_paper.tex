%%%%%%%%%%%%%%%%%%%%%%%%%%%%%%%%%%%%%%%%%%%%%%%%%%%%%%%%%%%%%%%%%%
%%%%%%%% ICML 2017 EXAMPLE LATEX SUBMISSION FILE %%%%%%%%%%%%%%%%%
%%%%%%%%%%%%%%%%%%%%%%%%%%%%%%%%%%%%%%%%%%%%%%%%%%%%%%%%%%%%%%%%%%

% Use the following line _only_ if you're still using LaTeX 2.09.
%\documentstyle[icml2017,epsf,natbib]{article}
% If you rely on Latex2e packages, like most moden people use this:
\documentclass{article}

% use Times
\usepackage{times}
% For figures
\usepackage{graphicx} % more modern
%\usepackage{epsfig} % less modern
\usepackage{subfigure} 

% For citations
\usepackage{natbib}

% For algorithms
\usepackage{algorithm}
\usepackage{algorithmic}

% As of 2011, we use the hyperref package to produce hyperlinks in the
% resulting PDF.  If this breaks your system, please commend out the
% following usepackage line and replace \usepackage{icml2017} with
% \usepackage[nohyperref]{icml2017} above.
\usepackage{hyperref}

% Packages hyperref and algorithmic misbehave sometimes.  We can fix
% this with the following command.
\newcommand{\theHalgorithm}{\arabic{algorithm}}

% Employ the following version of the ``usepackage'' statement for
% submitting the draft version of the paper for review.  This will set
% the note in the first column to ``Under review.  Do not distribute.''
%\usepackage{icml2017} 

% Employ this version of the ``usepackage'' statement after the paper has
% been accepted, when creating the final version.  This will set the
% note in the first column to ``Proceedings of the...''
\usepackage[accepted]{icml2017}


% The \icmltitle you define below is probably too long as a header.
% Therefore, a short form for the running title is supplied here:
\icmltitlerunning{Submission and Formatting Instructions for ICML 2017}

\begin{document} 


\twocolumn[
\icmltitle{Week of ---}

% It is OKAY to include author information, even for blind
% submissions: the style file will automatically remove it for you
% unless you've provided the [accepted] option to the icml2017
% package.

% list of affiliations. the first argument should be a (short)
% identifier you will use later to specify author affiliations
% Academic affiliations should list Department, University, City, Region, Country
% Industry affiliations should list Company, City, Region, Country

% you can specify symbols, otherwise they are numbered in order
% ideally, you should not use this facility. affiliations will be numbered
% in order of appearance and this is the preferred way.
\icmlsetsymbol{equal}{*}

\begin{icmlauthorlist}
%\icmlauthor{Aeiau Zzzz}{equal,to}
%\icmlauthor{Bauiu C.~Yyyy}{equal,to,goo}
%\icmlauthor{Cieua Vvvvv}{goo}
%\icmlauthor{Iaesut Saoeu}{ed}
%\icmlauthor{Fiuea Rrrr}{to}
%\icmlauthor{Tateu H.~Yasehe}{ed,to,goo} 
%\icmlauthor{Aaoeu Iasoh}{goo}
%\icmlauthor{Buiui Eueu}{ed}
%\icmlauthor{Aeuia Zzzz}{ed}
%\icmlauthor{Bieea C.~Yyyy}{to,goo}
%\icmlauthor{Teoau Xxxx}{ed}
%\icmlauthor{Eee Pppp}{ed}
\icmlauthor{Siddharthan Rajasekaran}{}
\end{icmlauthorlist}

%\icmlaffiliation{to}{University of Torontoland, Torontoland, Canada}
%\icmlaffiliation{goo}{Googol ShallowMind, New London, Michigan, USA}
%\icmlaffiliation{ed}{University of Edenborrow, Edenborrow, United Kingdom}
%
%\icmlcorrespondingauthor{Eee Pppp}{ep@eden.co.uk}

% You may provide any keywords that you 
% find helpful for describing your paper; these are used to populate 
% the "keywords" metadata in the PDF but will not be shown in the document
\icmlkeywords{boring formatting information, machine learning, ICML}

\vskip 0.3in
]

% this must go after the closing bracket ] following \twocolumn[ ...

% This command actually creates the footnote in the first column
% listing the affiliations and the copyright notice.
% The command takes one argument, which is text to display at the start of the footnote.
% The \icmlEqualContribution command is standard text for equal contribution.
% Remove it (just {}) if you do not need this facility.

%\printAffiliationsAndNotice{}  % leave blank if no need to mention equal contribution
\printAffiliationsAndNotice{\icmlEqualContribution} % otherwise use the standard text.

%\begin{abstract} 
%The purpose of this document is to provide both the basic paper template and
%submission guidelines. Abstracts should be a single paragraph, between 4--6 sentences long, ideally.  Gross violations will trigger corrections at the camera-ready phase.
%\end{abstract} 

\section{Summary of Discussions}
{\it Note: Use this section to quickly give context to this week's report.
This should give the key points from the previous discussion you had with me
and others (if relevant) and what is it that was decided as action
points/outcomes. If the discussions were a long time ago, you can summarize
the key points from the previous report here. Note that discussions could be
either in person or remotely.}

\section{Papers Read}
{\it Note: List out the papers you read in the last week. Even if you made
partial progress mention that here. For each paper, then give the reasons
that made you read that paper. This could be ``googled for keywords",
``referenced from another paper", ``wanted to learn more about xxx", etc.
Then, in your own words, give a summary of the contributions of the
paper/what you learnt from it. Also identify in what way the paper lacking
and if you have any ideas on how to improve it. This could involve applying
something from a different paper you read. Try to compare and contrast
other papers that tackle similar problems or use similar techniques. Cite
all appropriate papers.

I would recommend that researchers should pick up at least one new paper
every week and try to read it carefully. One has truly comprehended the
paper only when you can summarize the paper in your own words in 5-10
sentences. It is good practice to try and write down such a summary for
every paper that you read. It is an exercise to improve oneself, hence there
is no point in copying the abstract. List down strong and weak points of the
paper as you see it. Put down at least one take home message, even if it is
``do not write papers like this". :). Also write down questions that the
paper leaves unanswered, either by explicitly raising them, or by not
considering them. You can then try to find newer work that addresses these
questions. If none exist, you have potential problems to work on!! 

Maintain a bibtex file for all the paper that you read. Add to that every
week when you put down papers here. Do not trust the web to give you the
correct bibtex entry. Try to find out the details from the authors page or
journal/conference page and add it to the bibtex file. 

If you spend time revising some subjects you can mention that here, along
with the reasons for doing so. A detailed study plan can be mentioned in the
plan of action section.
}

\section{Theoretical Results}
{\it Note: Describe any theoretical results developed this week. This can
be algorithm specifications, experiment design, proof of theorems, statement
of theorems, etc. Describe clearly all the aspects involved - Background
results, why certain choices were made, what were the alternatives
considered, why they were rejected, etc. For example, if you decided to use
a L1-loss function, you should mention clearly what are the other options,
why you chose to use L1 over the others. If there are papers on which you
based these arguments, then cite them. Also add them to the bib file. }

\section{Experiments run}
{\it Note: Describe in as much detail as possible the experiments that you
set up and ran over this week. These could have been exploratory ones to
understand an algorithm/tool/package/domain/dataset/programming model, etc.
These could have been proof of concept experiments run on synthetic data or
small samples. The experiments could have succeeded or failed. But for each
of the experiments, you should write down clearly why you decided to run the
experiment. What was the set up? All the parameters used, no. of runs, etc.
Document all the outcomes clearly. Which parameter settings did the
experiment succeed, when did it fail. Why did you choose a certain setting?
Tabulate things clearly.. and where possible, draw graphs/charts etc.

Clearly document your code as well. Keep around different versions, with
comments describing what is different in each version. Try to follow some
coding standards. In experiments that take a long time to run, save as much
of the intermediate computation as possible. Do not write out only averages
to the file. This might take slightly longer to run initially, but will save
a lot of time when needing to re-run the experiments. For example, if you
want to do some feature computation and then run a classifier on some data,
save the computed features also. That way when you want to run a new
classifier you don't have to compute features again.}

\section{New Directions}

{\it Note: Put down any thoughts that you have on the problem. These might
be extensions to ideas that were discussed earlier. Might be ideas that came
from the papers that you read. Could be directions suggested by the outcome
of the experiments. The form could also be varied. These could be vague
ideas to serve as starting points for further discussions. These could be
concrete experiment designs. These could be ``theorems" that you would like
to try and prove, etc. Try to put in as much detail as possible, including
the thought process that lead you here. }

\section{Plan of Action}
{\it Note: List out all things that are planned for the near future, with
appropriate time lines. Ideally there should be several things for you to
work on next week, and some planned for the subsequent weeks and months.
Again, try to be as detailed as possible. Make references to the text above,
do not repeat unnecessarily. Do not say run new experiments. Say ``Set up
experiments described in Section \ ref\{new-section\}, in order to validate
hypothesis 3" or something similar. These action points might be recounted
in the next weeks report. 

Importantly, identify conferences that you plan to communicate the work you
are currently engaged in. Find out the submission deadlines and plan working
backwards. ``I have 8 weeks to work and this is what I can do each week." Plan
on at least a week for writing. Often our work gets rejected due to bad
presentation.}

\section{Random Thoughts}
{\it Note: Put down things that you do not think fit into the framework
above, but those that you want to record for my information, or for future
reference. }

\section{Bibliography}
{\it Note:} Ideally this shd be generated by bibtex and not manually by
you.\\
 \\

{\it {\bf Other Guidelines:} You do not have to populate each and every section
every week. But I expect at least one of the sections to have significant
material. If you receive any help from any one be generous in
acknowledgements. These reports will help immensely when it comes to writing
papers or at the very least when it comes to final thesis/reports. Mention
your name clearly at the top. Use LaTeX. Name the file as follows:
``WeeklyReport\_yourname\_ddmmyy.pdf", where ddmmyy is the date of the Monday
of the corresponding week.}

\nocite{langley00}

\bibliography{example_paper}
\bibliographystyle{icml2017}

\end{document} 


% This document was modified from the file originally made available by
% Pat Langley and Andrea Danyluk for ICML-2K. This version was
% created by Lise Getoor and Tobias Scheffer, it was slightly modified  
% from the 2010 version by Thorsten Joachims & Johannes Fuernkranz, 
% slightly modified from the 2009 version by Kiri Wagstaff and 
% Sam Roweis's 2008 version, which is slightly modified from 
% Prasad Tadepalli's 2007 version which is a lightly 
% changed version of the previous year's version by Andrew Moore, 
% which was in turn edited from those of Kristian Kersting and 
% Codrina Lauth. Alex Smola contributed to the algorithmic style files.  
